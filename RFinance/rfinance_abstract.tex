\documentclass[12pt]{article}    
\usepackage{times,graphicx,fancyhdr,amsfonts,amsthm,amsmath,xspace,verbatim,enumerate,listings,multicol,multirow}
\usepackage{booktabs} % nice tables
%\usepackage{tikz,tikz-qtree}
\usepackage[left=1in,top=1in,right=1in,bottom=1in]{geometry}
\usepackage[sc]{mathpazo}
\linespread{1.2}         % Palatino needs more leading (space between lines)
\usepackage[T1]{fontenc}
\newcommand{\head}[1]{\textnormal{\textbf{#1}}}
\newcommand{\ra}[1]{\renewcommand{\arraystretch}{#1}}


\newenvironment{benumerate}
  {\begin{enumerate}[a]\renewcommand\labelenumi{\textbf\theenumi .}}
  {\end{enumerate}}

\title{Project Abstract for R/Finance 2014}
\author{Heidi Chen\thanks{s.heidi.chen@gmail.com}, David Kane\thanks{dave.kane@gmail.com}, and Yang Lu\thanks{yang.lu2014@gmail.com}}


\lstset{
   numbers=left,
   numberstyle=\small,
   breakatwhitespace=true,
   breaklines=true
}

\theoremstyle{plain}
\newtheorem{thm}{Theorem}[section]

\newtheorem{lem}[thm]{Lemma}

\begin{document}
\vspace*{-3cm}
 {\let\newpage\relax\maketitle}
%\maketitle

%\thispagestyle{fancy}

The \textbf{CDS} package offers tools to calculate the value of a
Credit Default Swap (CDS). A CDS is a financial swap agreement
between two counterparties where the buyer pays a fixed periodic
coupon to the seller in exchange for protection of an occurrence of a
credit event. The International Swaps and Derivatives Association
(ISDA) has created a set of standard terms for the new CDS
contracts. These new contracts are required to trade at a standardized
fixed coupon rate (100/500 bps) with full first coupon. The ISDA has
also provided the Standard Model to calculate cash settlement from
conventional spread, convert between conventional spread and upfront
payment, and build the yield curve of a CDS. The \textbf{CDS} package
implements the ISDA Standard Model, allowing users to value both newly
created and existing CDS contracts and compute their rates of returns.



\end{document}
